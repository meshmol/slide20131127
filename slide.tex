\documentclass[dvipdfm,12pt,fleqn]{beamer}
\usepackage{bm,url,ascmac,amssymb,amsmath,graphicx}
\usetheme{Warsaw}
\usefonttheme[onlymath]{serif}
\renewcommand{\kanjifamilydefault}{\gtdefault}
\setbeamertemplate{navigation symbols}{}

\title{What Does R7RS\\ Change Programming in Scheme?}
\author[Y. Kurosaki \and K. Hishinuma]{Yuta Kurosaki \and Kazuhiro Hishinuma}
\institute[Department of Computer Science, Meiji University]{%
Department of Computer Science, Meiji University\\%
 1-1-1 Higashimita, Tama-ku, Kawasaki-shi, Kanagawa, 214-8571 Japan}
\date{}

\begin{document}
\begin{frame}
\titlepage
\end{frame}

\begin{frame}{About Speakers}
\begin{Huge}
Yuta Kurosaki
\end{Huge}\\
Web Science Laboratory, Meiji University.

\begin{itemize}
\item A developer of Meiji Scheme Shell
\item IPA Security \& Programming Camp 2010 OB\\
(Operation system class, programming course)
\item Twitter: @kuro\_m88
\end{itemize}
\end{frame}

\begin{frame}{About Speakers}
\begin{Huge}
Kazuhiro Hishinuma
\end{Huge}\\
Mathematical Optimization Laboratory, Meiji University.

\begin{itemize}
\item Shibuya.lisp Member
\item Interested in mathematics and Scheme
\item Twitter: @kazh98
\end{itemize}
\end{frame}

\begin{frame}{What is the Schemers' Soul?}
\huge
Snap out of it, Schemers!
\pause

\vspace{1em}
\alert{Scheme} is the simplest,\\
the smallest, and\\
the most powerful language!
\end{frame}

\begin{frame}{What is the Programmers' Utopia?}
\huge
And..., join us,\\
\alert{all lispers} and programmers!
\pause

\vspace{1em}
Now, \alert{the most ideal language}\\
is going to be born!
\end{frame}

\begin{frame}{Congratulation!}
\pause\Huge
R7RS-small draft\\
\alert{ratified} \Large{}by Steering Committee!!
\footnote{\url{http://lists.scheme-reports.org/pipermail/scheme-reports/2013-November/003832.html}}
\end{frame}

\begin{frame}{R$^7$RS says ...}
\Large
``Scheme demonstrates that\\
\pause \alert{a very small number of rules}\\
for forming expressions,\pause with no restrictions\\
\pause \alert{on how they are composed}."
\end{frame}

\begin{frame}{So, today.}
\huge
Let us think what is\\
\alert{The Genuine Programming}\\
In R$^7$RS Scheme.
\end{frame}

% TOPIC1: Record-type pp.27--
% TOPIC2: Library System pp.28--
% TOPIC3: Exceptions pp.54--
% TOPIC4: Other Changes
%   Case sensitivity is now the default in symbols and character names.
%   Case-lambda pp.21--
%   The call-with-current-continuation procedure now has the synonym call/cc.

\begin{frame}
\Huge
Think in Scheme,\\
write in Scheme,\\
and show \alert{your Scheme}!

\vspace{1em}
\begin{flushright}
\Large
Thanks for your listening.
\end{flushright}
\end{frame}

\begin{frame}{References}
\footnotesize
\begin{thebibliography}{9}
\beamertemplatetextbibitems
\bibitem{email} J. Cowan: \textbf{R7RS-small draft ratified by Steering Committee}. The public mailing lists on \url{lists.scheme-reports.org}, 2013. \url{http://lists.scheme-reports.org/pipermail/scheme-reports/2013-November/003832.html}
\bibitem{r7rs} A. Shinn, J. Cowan, and A. Gleckler: \textbf{Revised$^7$ Report on the Algorithmic Language Scheme}. Steering Committee, Scheme Working Groups, 2013. \url{http://trac.sacrideo.us/wg/}
\bibitem{mesh} Y. Kurosaki, and K. Hishinuma: \textbf{Meiji Scheme Shell improved by MOL}. Meiji Scheme Project, Mathematical Optimization Laboratory, Meiji University. \url{https://github.com/meshmol/mesh}
\bibitem{scm} K. Sasagawa: \textbf{Normal Scheme}. Scheme, 2013. \url{http://homepage1.nifty.com/~skz/Scheme/normal.html}
\end{thebibliography}
\end{frame}

\end{document}
