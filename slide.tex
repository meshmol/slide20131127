\documentclass[dvipdfm,12pt,fleqn]{beamer}
\usepackage{bm,url,ascmac,amssymb,amsmath,graphicx,multicol}
\usetheme{Warsaw}
\usefonttheme[onlymath]{serif}
\renewcommand{\kanjifamilydefault}{\gtdefault}
\setbeamertemplate{navigation symbols}{}

\title{What Does R7RS\\ Change Programming in Scheme?}
\author[K. Hishinuma]{Kazuhiro Hishinuma (Twitter: @kazh98)}
\institute[Department of Computer Science, Meiji University]{%
Department of Computer Science, Meiji University\\%
 1-1-1 Higashimita, Tama-ku, Kawasaki-shi, Kanagawa, 214-8571 Japan}
\date{}

\begin{document}
\begin{frame}
\titlepage
\end{frame}

\begin{frame}{What is the Schemers' Soul?}
\huge
Snap out of it, Schemers!
\pause

\vspace{1em}
\alert{Scheme} is the simplest,\\
the smallest, and\\
the most powerful language!
\end{frame}

\begin{frame}{What is the Programmers' Utopia?}
\huge
And..., join us,\\
\alert{all lispers} and programmers!
\pause

\vspace{1em}
Now, \alert{the most ideal language}\\
is going to be born!
\end{frame}

\begin{frame}{Congratulation!}
\pause\Huge
R7RS-small draft\\
\alert{ratified} \Large{}by Steering Committee!!
\footnote{\url{http://lists.scheme-reports.org/pipermail/scheme-reports/2013-November/003832.html}}
\end{frame}

\begin{frame}{R$^7$RS says ...}
\Large
``Scheme demonstrates that\\
\pause \alert{a very small number of rules}\\
for forming expressions, \pause with no restrictions\\
\pause \alert{on how they are composed}."
\end{frame}

\begin{frame}{So, today.}
\huge
Let us think what is\\
\alert{The Genuine Programming}\\
In R$^7$RS Scheme.
\end{frame}

\begin{frame}{Three Hot Changes\footnotemark[1]}
\huge
\begin{itemize}
\item Record-type (cf. pp.27--)
\item Library System (cf. pp.28--)
\item Exceptions (cf. pp.54--)
\end{itemize}

\footnotetext[1]{from R$^5$RS}
\end{frame}

\begin{frame}{Three Hot Changes\footnotemark[1]}
\huge
\begin{itemize}
\item \alert{Record-type} (cf. pp.27--)
\item Library System (cf. pp.28--)
\item Exceptions (cf. pp.54--)
\end{itemize}

\footnotetext[1]{from R$^5$RS}
\end{frame}

\begin{frame}{Record-type}
\begin{screen}
(\textbf{define-record-type} \textit{name}\\
\hspace{1em}(\textbf{\textit{cname}} \textit{f$_1$} \textit{f$_2$} ...)\\
\hspace{1em}\textit{pred?}\\
\hspace{1em}(\textit{f$_1$} \textit{ref-f$_1$} \textit{set-f$_1$!})\\
\hspace{1em}(\textit{f$_2$} \textit{ref-f$_2$} \textit{set-f$_2$!})\\
\hspace{1em}...)
\end{screen}
\begin{description}
\item[name] Name of the record to be defined
\item[pred?] Name of the predicatior for this record
\item[f$_1$, f$_2$, ...] Names of the fields of this record
\end{description}
\end{frame}

\begin{frame}[containsverbatim]{e.g. CONS, CAR, CDR}
\textbf{Existing method:}
\begin{screen}
(\textbf{define} (\textbf{cons} a b)\\
\hspace{1em}(\textbf{lambda} (s) (\textbf{s} a b)))\\
(\textbf{define} (\textbf{car} c)\\
\hspace{1em}(\textbf{c} (\textbf{lambda} (a b) a)))\\
(\textbf{define} (\textbf{cdr} c)\\
\hspace{1em}(\textbf{c} (\textbf{lambda} (a b) b)))\\
\end{screen}

\begin{multicols}{2}
\begin{verbatim}
(define c (cons 'a 'b))
(car c) ;=> 'a
(cdr c) ;=> 'b
(pair? c) ;=> ?!
(set-car! c 'd)
(car c) ;=> ?!
\end{verbatim}
\end{multicols}
\end{frame}

\begin{frame}[containsverbatim]{e.g. CONS, CAR, CDR}
\textbf{Proposed method:}
\begin{screen}
(\textbf{define-record-type} $<$pair$>$\\
\hspace{1em}(\textbf{cons} a b)\\
\hspace{1em}pair?\\
\hspace{1em}(a car set-car!)\\
\hspace{1em}(b cdr set-cdr!))
\end{screen}

\begin{multicols}{2}
\begin{verbatim}
(define c (cons 'a 'b))
(car c) ;=> 'a
(cdr c) ;=> 'b
(pair? c) ;=> #t
(set-car! c 'd)
(car c) ;=> 'd
\end{verbatim}
\end{multicols}
\end{frame}

\begin{frame}{define-record-type give Scheme ...}
\Huge\pause
The seed of\\
\alert{Object-Oriented} Programming!
\end{frame}

\begin{frame}{Three Hot Changes\footnotemark[1]}
\huge
\begin{itemize}
\item Record-type (cf. pp.27--)
\item \alert{Library System} (cf. pp.28--)
\item Exceptions (cf. pp.54--)
\end{itemize}

\footnotetext[1]{from R$^5$RS}
\end{frame}

\begin{frame}{Good news!}
\huge
Notation for library system\\
is \alert{standardized}!!
\end{frame}

\begin{frame}{How to load SRFI-1}
\Huge
\begin{description}
\item[Gauche] (\textbf{use} srfi-1)
\item[Guile] (\textbf{srfi} srfi-1)
\item[Racket] (\textbf{require} srfi/1)
\pause
\item[R$^7$RS] (\alert{\textbf{import}} srfi-1)
\end{description}
\end{frame}

\begin{frame}{How to make a library}
\begin{screen}
(\textbf{define-library} \textit{name}\\
\hspace{1em}(\textbf{export} \textit{ep} ...)\\
\hspace{1em}(\textbf{define} (p1 args ...)\\
\hspace{2em}...) ...)\\
\end{screen}

\begin{description}
\item[name]Name of the library
\item[ep ...]List of names to be exported
\end{description}
\end{frame}

\begin{frame}{Three Hot Changes\footnotemark[1]}
\huge
\begin{itemize}
\item Record-type (cf. pp.27--)
\item Library System (cf. pp.28--)
\item \alert{Exceptions} (cf. pp.54--)
\end{itemize}
\end{frame}

\begin{frame}{What's this?}
\Huge
SRFI-34 is \alert{included}\\
in R$^7$RS!
\end{frame}

\begin{frame}{with-exception-handler}
e.g.
\begin{screen}
(\textbf{with-exception-handler}\\
\hspace{1em}(\textbf{lambda} (e) \textit{Handler for Exception e})\\
\hspace{1em}(\textbf{lambda} () \textit{Procedure which may raise exception}))
\end{screen}
\begin{screen}
(\textbf{error} "This is an error message.")
\end{screen}
\end{frame}

\begin{frame}{Other Changes}
\Large
\begin{itemize}
\pause\item Case sensitivity is now the default in symbols and character names.
\pause\item Case-lambda (cf. pp.21--)
\pause\item The call-with-current-continuation procedure now has the synonym call/cc.
\end{itemize}
\end{frame}

\begin{frame}
\Huge
Think in Scheme,\\
write in Scheme,\\
and show \alert{your Scheme}!

\vspace{1em}
\begin{flushright}
\Large
Thanks for your listening.
\end{flushright}
\end{frame}

\begin{frame}{References}
\footnotesize
\begin{thebibliography}{9}
\beamertemplatetextbibitems
\bibitem{email} J. Cowan: \textbf{R7RS-small draft ratified by Steering Committee}. The public mailing lists on \url{lists.scheme-reports.org}, 2013. \url{http://lists.scheme-reports.org/pipermail/scheme-reports/2013-November/003832.html}
\bibitem{r7rs} A. Shinn, J. Cowan, and A. Gleckler: \textbf{Revised$^7$ Report on the Algorithmic Language Scheme}. Steering Committee, Scheme Working Groups, 2013. \url{http://trac.sacrideo.us/wg/}
\bibitem{mesh} Y. Kurosaki, and K. Hishinuma: \textbf{Meiji Scheme Shell improved by MOL}. Meiji Scheme Project, Mathematical Optimization Laboratory, Meiji University. \url{https://github.com/meshmol/mesh}
\bibitem{scm} K. Sasagawa: \textbf{Normal Scheme}. Scheme, 2013. \url{http://homepage1.nifty.com/~skz/Scheme/normal.html}
\end{thebibliography}
\end{frame}

\end{document}
